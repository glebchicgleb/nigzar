\documentclass{article}
\usepackage[russian]{babel}
\usepackage{graphicx} % Required for inserting images

\title{Мини-конспект по теме: Теорема Пифагора}
\author{Воложев Глеб}
\date{26 сентября 2025 г.}

\begin{document}

\maketitle

\tableofcontents
\newpage

\section{Введение}
Теорема Пифагора — одна из важнейших теорем евклидовой геометрии. Она находит применение в самых разных областях:
\begin{itemize}
  \item геометрия и тригонометрия
  \item физика
  \item инженерные расчёты
  \item компьютерная графика
\end{itemize}

\section{Формулировка теоремы}
Слова: В прямоугольном треугольнике квадрат гипотенузы равен сумме квадратов катетов.
\begin{equation}
c^2 = a^2 + b^2
\end{equation}
Как видно из формулы (1), знание двух сторон позволяет найти третью.

\section{Доказательство (набросок)}
Одно из доказательств основывается на площади квадрата, составленного из четырёх одинаковых прямоугольных треугольников и малого квадрата в центре.  
Раскладывая площадь двумя способами, получаем $c^2 = a^2 + b^2$.

\section{Примеры расчёта}

\textbf{Пример 1}  

$a = 3, \; b = 4$  

\[
c = \sqrt{a^2 + b^2} = \sqrt{9 + 16} = 5
\]

\textbf{Пример 2}  

1. Дано: $a = 5, \; b = 12$  

2. Решение:  
\[
c = \sqrt{5^2 + 12^2} = \sqrt{25 + 144} = 13
\]

\section{Таблица значений}

\begin{center}
\begin{tabular}{|c|c|c|}
\hline
Катет $a$ & Катет $b$ & Гипотенуза $c$ \\
\hline
3 & 4 & 5 \\
5 & 12 & 13 \\
7 & 24 & 25 \\
\hline
\end{tabular}
\end{center}

\section{Иллюстрация}

\begin{center}
\includegraphics[width=0.5\textwidth]{triangle.png}
\end{center}

\section{Заключение}
Теорема Пифагора — один из краеугольных камней геометрии, помогающий решать множество практических задач.

\section{Ссылки и литература}
\begin{itemize}
  \item \href{https://ru.wikipedia.org/wiki/Теорема_Пифагора}{Википедия: Теорема Пифагора}
  \item Классические учебники геометрии
\end{itemize}

\end{document}
